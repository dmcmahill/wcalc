%

\documentclass[11pt]{article}
\usepackage{graphicx}
\usepackage[ps2pdf,breaklinks=true,colorlinks=true]{hyperref}

\setlength{\topmargin}{0in}
\setlength{\textheight}{8.0in}
\setlength{\evensidemargin}{-0.25in}
\setlength{\oddsidemargin}{-0.25in}
\setlength{\textwidth}{7in}


\title{Coaxial Transmission Line Waveguide Modes}
\author{Dan McMahill}
\date{February 2009}

\usepackage{fancyheadings}
\pagestyle{fancy}
\rhead{Coaxial Transmission Line Waveguide Modes}
\chead{}
\lhead{Dan McMahill}
\cfoot{\thepage}
\lfoot{\tiny{Last rev on \today}}
\rfoot{\tiny{\copyright 2009 Dan McMahill}}

\begin{document}
\maketitle

\section{Overview}

\section{Review of Maxwells Equations}

Gauss' law for magnetic fields and electric fields are:
\begin{equation}
\nabla \cdot \vec{H} = 0
\end{equation}
\begin{equation}
\nabla \cdot \vec{D} = \rho
\end{equation}
The magnetic field is $\vec{H}$.  The electric flux density, $\vec{D}$, is related
to the electric field, $\vec{E}$, and the permitivity of the medium by
\begin{equation}
\vec{D} = \epsilon \vec{E}
\end{equation}
Finally, the charge density is $\rho$.  Simply stated, the integral of
the magnetic field normal to a closed surface is zero.  What goes in,
comes out.  The integral of the electric flux density normal to a
closed surface is equal to the enclosed charge.

The other half of Maxwell's equations are Faraday's law,
\begin{equation}
\nabla \times \vec{E} = - \frac{d \vec{B}}{d t}
\end{equation}
and Ampere's law.
\begin{equation}
\nabla \times \vec{H} = \vec{J} + \frac{d \vec{D}}{d t}
\end{equation}

Faraday's law tells us that the electric field is influenced by the
time rate of change of the magnetic flux density, $\vec{B}$.  In particular,
the integral of the electric field around a closed path is the same as
the time rate of change of the integrated magnetic flux density normal
to any surface defined by the closed path.

Ampere's law tells us that the curl of the magnetic field is given by
the current density and the time derivatve of the electric flux density.  In
integral form Ampere's law says that the integral of the magnetic field
along a closed path equals the integral of the current density through
the surface defined by that path plus the time rate of change of the
electric flux density through the surface defined by the path.

For the purposes of wave propagation, it is useful to write the
electric and magnetic fields as the product of a function that only
varies with spacial coordinates and another function that only varies
with time.
\begin{equation}
\vec{H}(t,x,y,z) = \vec{H}(x,y,z) \mathrm{e}^{j \omega t}
\end{equation}
\begin{equation}
\vec{E}(t,x,y,z) = \vec{E}(x,y,z) \mathrm{e}^{j \omega t}
\end{equation}

\subsection{The Wave Equation}

\begin{equation}
\nabla_t^2 H_z + k^2 H_z = 0
\end{equation}

In cylindrical coordinates, the operator $\nabla_t^2$ is
\begin{equation}
\nabla_t^2 = \frac{1}{\rho} \frac{\partial}{\partial \rho}
\left(\rho \frac{\partial}{\partial \rho} \right) + 
\frac{1}{\rho^2}\frac{\partial^2}{\partial \phi^2}
\end{equation}
The wave equation then becomes
\begin{equation}
\frac{1}{\rho} \frac{\partial}{\partial \rho}
\left(\rho \frac{\partial H_z\left(\rho, \phi\right)}{\partial \rho} \right) + 
\frac{1}{\rho^2}\frac{\partial^2 H_z\left(\rho, \phi\right)}{\partial
  \phi^2} + k^2 H_z\left(\rho, \phi\right) = 0
\end{equation}
The solution is well known, see \cite{terman} for example, and is
given by
\begin{equation}
\end{equation}

All that remains is to pick the constants to satisfy the boundary
conditions imposed by our structure.  For a circular waveguide, a
structure covered by many textbooks, the boundary condition is that
the tangential component of the electric field must be zero at the
surface of the ideal conductor.  In addition, the fields at the origin
($\rho = 0$) must be continuous.  This latter constraint eliminates
the ?? terms from our candidate solution.  For the case of a coaxial
line, our boundary condition at the shield is identical to the
circular waveguide case.  However we have traded off the condition of
continuity at $\rho = 0$ for the condition that the tangential
component of the electric field must be zero at the surface of the
inner conductor.



\appendix
\cite{terman}
%

%\bibliographystyle{apalike}
\bibliographystyle{ieeetr}
\bibliography{refs}
\thispagestyle{fancy}


\end{document}








